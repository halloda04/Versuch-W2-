\section{Versuchsbeschreibung}

\subsection{Versuchsaufbau}
Für den Versuchsaufbau werden an Geräten benötigt: ein 
Heizpilz mit passendem Rundkolben, Siedesteinchen, ein 
Dichtring mit einem Verbindungspropfen, zwei Gefäße wie in 
<<<<<<< HEAD
Abbildung 2 mit den dazugehörigen Dichtungen, um 
sie miteinander zu verbinden, aber auch um sie mithilfe des 
=======
der Abbildung \ref{Gefäß3} mit den dazu gehörigen Dichtungen um 
sie miteinander zuverbinden aber auch um sie mithilfe des 
>>>>>>> 21829b2d1073189c5eaec0172fbaa0b2299cbb6d
Thermometers geschlossen zu halten, ein Thermometer, eine Waage, zwei 
Schläuche und eine Schlauchklemme. Außerdem werden verschiedene 
Lösungsmittel benötigt. Empfehlenswert hierfür sind destilliertes 
Wasser, eine verdünnte Harnstofflösung und zwei NaCl-Lösungen, 
eine mit 10\% NaCl-Gehalt und die andere mit 20\%.  
Außerdem sind Befestigungen für alles nötig, damit jedes Teil 
im Aufbau ohne großen Aufwand verstellbar ist.
Nun zum Aufbau: Hierzu wird der Rundkolben mit dem Heizpilz 
befestigt und mit 150–200 Millilitern destilliertem Wasser und 
den Siedesteinchen gefüllt. Daraufhin wird Gefäß 2 aus der 
Abbildung leer gewogen, mit 40 Millilitern des jeweiligen 
Lösungsmittels gefüllt und erneut gewogen. Sobald dies 
geschehen ist, wird Gefäß 2 in Gefäß 1 eingesetzt und so 
luftdicht verschlossen, dass Öffnung a unterhalb der Dichtung 
noch innerhalb des Gefäßes liegt. Dieses zusammengesetzte 
neue Gefäß wird nun in den Rundkolben eingesetzt und erneut 
dicht verschlossen. Um das Gefäß nach oben hin luftdicht zu verschließen, 
wird dort nun auch die Dichtung angebracht und mithilfe des 
Thermometers komplett abgedichtet. Dabei muss darauf geachtet 
werden, dass das Thermometer nicht in die Flüssigkeit in Gefäß 
2 hineinragt, aber auch nicht das Glasstück zwischen Punkt a 
und b berührt. Als Letztes werden nun die zwei Schläuche an 
den beiden Öffnungen angebracht, sodass mögliches Kondenswasser 
gesammelt werden kann. Nun wird die Schlauchklemme an 
dem Schlauch befestigt, der an der Öffnung des Gefäßes 1 
angebracht wurde.

\begin{figure}[H]
    \centering
    \includegraphics[scale = 0.8]{Bilder/Bild Gefäße.png}
    \label{Gefäß3}
    \caption{Gefäße 1 und 2}
\end{figure}
\newpage

\subsection{Versuchsdurchführung}
Sobald der Versuchsaufbau vollendet wurde, kann der Heizpilz 
eingeschaltet werden. Sobald am Thermometer die Temperatur 
80\,°C angezeigt wird, wird der Heizpilz vom Rundkolben 
abgesenkt, sodass das Wasser im Kolben aufhört zu sieden und 
das kondensierte Wasser innerhalb des zusammengesetzten Gefäßes 
wieder zurück in den Rundkolben fließt. Wenn das Wasser wieder 
zurückgeflossen ist, kann der Heizpilz wieder angehoben werden. 
Nun wird gewartet, bis die Temperatur am Thermometer um circa 
4\,°C gestiegen ist. Sobald dies geschehen ist, wird die 
Schlauchklemme geschlossen und gewartet, bis die Lösung in 
Gefäß 2 anfängt zu sieden und sich die Temperatur nicht mehr 
ändert. Diese maximale Temperatur wird nun festgehalten, 
und die Schlauchklemme wird wieder geöffnet und der Heizpilz 
gesenkt und ausgeschaltet. Hierbei ist es wichtig, darauf zu 
achten, dass zuerst die Schlauchklemme geöffnet wird und 
danach erst der Heizpilz abgesenkt wird. Als letzter Teil der 
Durchführung wird nun Gefäß 2 ausgebaut und wieder gewogen. 
Dabei muss jedoch darauf geachtet werden, dass alle Glasgefäße 
noch stark erwärmt sind und somit eine Verbrennungsgefahr 
besteht. Diese Schritte werden nun für die anderen Lösungen wiederholt.

