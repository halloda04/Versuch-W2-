\section{Auswertung}

\subsection{Teil 1, ebullioskopische Konstante}

it Hilfe der Gleichung \ref{eq:K} kann die ebullioskopische Konstante bestimmt werden. Für die Siedetemperatur des destilliertem Wasser wurde eine Temperatur von
$98,6 \pm 0,05°C$ gemessen. Mit den Konstanten entnommen aus [Quellen] ergibt sich für $K$

\begin{equation}
K = (0,509 \pm 0,0025)\frac{K \cdot kg}{mol}
\end{equation}

Wobei der Fehler durch:

\begin{equation}
    \Delta K = \pm \left( \left| \frac{\text{\partial K}}{\text{\partial} T_0} \Delta T_0 \right| \right) = 
    \pm \frac{2M_A R T_0 \Delta T_0}{\Lambda_A}
\end{equation}
gegeben ist. Die Literaturwerte als Fehlerfrei angenommen. Für den Fehler der Temperatur wurde der Fehler des Thermometers welcher sich bei $100°C$ auf etwa $\pm 0,9°C$ beläuft mit dem Ablesefehler von $\pm0,1°C$ addiert.  Auch wenn sich der Literaturwert von $0,513$ knapp nicht in der Fehlerschranke befindet, ist der relative Fehler von $0,8 \%$ recht gering und die Bestimmung der ebullioskopische Konstante kann als erfolgreich gewertet werden.


\subsection{Teil 2, Konzentration Harnstoff}
Für den Siedepunkt des Harnstoffes wurde ein Wert von $T_H$ von $99,2 \pm 0,05°C$ gemessen. Es befanden sich, vor dem Kochvorgang
$38,2  \pm 0,2$g Harnstofflösung in dem Gefäß. Der Fehler hier ist durch Summe der Fehler (Des leeren  und des vollen Gefäßes) gegeben. Durch den Siedeprozess wurden $0,5 \pm 0,2$g Wasser eingeschleppt. Der Fehler berechnet sich hier genauso wie bei der Harnstofflösung. Die Kozentration $C$ welche durch

\begin{equation}
C = \frac{m_B}{m_A}
\end{equation}

gegeben ist. Wobei $m_B$ die Masse des Harnstoffes und $m_A$ die ursprüngliche Menge an Wasser im Glas beschreibt.$M_B$ ist die Molaremasse des Harnstoffes. $m_e$ ist die Masse des eingeschleppten Wassers. Es gilt nach Gleichung \ref{eq:dT}

\begin{equation}
m_{Ae} = \frac{K \cdot m_{B}}{\Delta T  M_B} - m_e
\end{equation}
Mit

\begin{equation}
m_{nachher} = m_{vorher} + m_e
\end{equation}

Setzt man dies in 
\begin{equation}
m_{B} = m_{vorher} - m{A}
\end{equation}

ein. Erhält man
\begin{equation}
m_B =m_{vorher} - \frac{K \cdot m_{B}}{\Delta T  M_B} - m_e
\end{equation}




\begin{equation}
m_b = \frac{(\Delta T \cdot M_B) (\cdot m_{vorher} -m_e)}{K}
\end{equation}

Das ergibt für die Konzentration

\begin{equation}
C = \frac{(\Delta T \cdot M_B) (\cdot m_{vorher} -m_e)}{K \cdot m_{vorher}} = \frac{\Delta T \cdot M_B}{K} \left( 1 - \frac{m_e}{m_\text{vorher}} \right) = (7,1 \pm 10,7)\%
\end{equation}

Der Fehler der Konzentration ist gegeben durch

\begin{equation}
\begin{split}
\Delta C &= \pm \left| \frac{\partial C}{\partial \Delta T} \right| \cdot \Delta(\Delta T) 
+ \left| \frac{\partial C}{\partial m_e} \right| \cdot \Delta m_e 
+ \left| \frac{\partial C}{\partial m_\text{vorher}} \right| \cdot \Delta m_\text{vorher} \\
&=\pm \frac{M_B}{K} \Bigg[ 
\left( 1 - \frac{m_e}{m_\text{vorher}} \right) \Delta (\Delta T) 
+ \Delta T \left( \frac{\Delta m_e}{m_\text{vorher}} + \frac{m_e}{m_\text{vorher}^2} \Delta m_\text{vorher} \right)
\Bigg]
\end{split}
\end{equation}

da die Literaturwerte als nicht Fehlerbehaftet angesehen wurden. Dies ist ein sehr signifikanter Fehler von $150\%$ Was vorallem von den Fehler der Siedepunkterhöhung kommt, welcher wiederum so groß ist, da das verwendete Thermometer sehr ungenau misst. Zudem ist die Näherung $ \frac{p}{p_0} \approx \frac{n_B}{n_A}$ welche für die Herleitung der Gleichung \ref{eq:dT} verwendet wurde hier  nicht mehr gültig da mit einem Masseanteil von $7,1 \%$ nicht mehr $m_A$ \gg $m_B$ gilt.

\subsection{Teil 3, Dissoziationsgrad}

Mit Gleichung \ref{eq:alpha} lässt sich der Dissoziationsgrad der Natriumchloridlösungen bestimmen. Es wurden zwei Messungen durchgeführt, einem mit einer Konzentration von 10\% und eimal mit 20\%. Die Masse des 10\%-igen NaCL = $m_B = C \cdot m_{vorher} = (4,05 \pm 0,2)g$ wobei sich $m_v = (40,5 \pm 0,2)g$ aus der Masse des leeren Gefäßes und des gefüllten Gefäßes ergibt. Die Masse des Lösungsmittels samt eingeschlepptem Wasser kann mit

\begin{equation}
m_A = (1 - C)\cdot m_{vorher} + m_{nacher} - m{vorher} = 37,05 FEHLER
\end{equation}

berechnet werden. Der Dissozieationsgrad ist somit durch

\begin{equation}
\alpha = \frac{\frac{\Delta T m_A \cdot M_B}{K m_B}-1}{z-1}
\end{equation}
