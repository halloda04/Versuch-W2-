\section{Theoretische Grundlagen}

\subsection{Aggregatzustände}
Die drei Aggregatzustände, die wir in unserem täglichen Leben 
beobachten können, sind gasförmig, flüssig und fest. Im 
gasförmigen Zustand haben die Teilchen eines Stoffes genug 
Energie, um die Wechselwirkungen, die sie untereinander haben, 
zu durchbrechen; dadurch können sich die Teilchen frei im 
Raum bewegen. Wenn nun Energie von diesen Teilchen entfernt 
wird, zum Beispiel durch Abkühlen, werden diese Teilchen 
zuerst zu einer Flüssigkeit. Wenn weiter Energie entzogen 
wird, gefriert diese Flüssigkeit und wird zu einem Feststoff. 
Teilchen in einer Flüssigkeit haben noch genug Energie, um 
sich zu bewegen, aber nicht mehr genug, um sich 
vollständig frei zu bewegen. In einem Feststoff wiederum ist so 
wenig Energie vorhanden, dass die Teilchen oft in einem Gitter 
angeordnet vorzufinden sind und sich nur noch minimal durch 
Schwingungen bewegen können.

\begin{figure}[H]
    \centering
    \includegraphics[scale = 0.5]{Bilder/Phasendiagramm.png}
    \caption{Das p-T Diagramm zu einem beliebigen Stoff mit drei Phasen\cite{Phasendiagramm}}
    
\end{figure}

In diesem Diagramm sind die Übergänge dieser drei 
Aggregatzustände in Abhängigkeit von ihrer 
Temperatur und dem Druck des betrachteten Volumen zu sehen. 

\newpage
\subsection{Dampfdruck}
Die Dampfdruck Kurve bezeichnet den Übergang vom Flüssigen 
in das Gasförmige. Genau auf der Dampfdruckkurve befindet 
sich ein Thermodynamisches Gleichgewicht zwischen Teilchen 
die aus der Flüssigkeit in das Gasförmige übergehen und 
umgekehrt
\subsection{Siedepunkterhöhung}
Mit mehr bzw stärkeren molekularen Wechselwirkungskräften kann der Siedepunkt erhöht werden. Dies kann erreicht werden indem man
eine Substanz in dem Lösungsmittel löst und somit neue intermolekulare Bindungen in der Lösung entstehen. Diese Siedepunkterhöhung 
kann mit Folgender Formel ausgedrückt werden.

\begin{equation}
    \Delta T = \frac{m_B}{m_A M_B}K
    \label{eq:dT}
\end{equation}

Dabei sind $m_A$ und $m_B$ die Massen des Lösungsmittels bzw. der dazugegebenen Substanz. $M_A$ ist die Modularemasse der gelösten Substanz Die Konstante K hängt vom Lösungsmittel ab und wird als
ebullioskopische Konstante bezeichnet, sie ist gegeben durch:

\begin{equation}
    K = \frac{M_A R T_0^2}{\Lambda_A}
    \label{eq:K}
\end{equation}

Wobei $M_A$ die Molaremasse des Lösungsmittels, $R$ die allgemeine Gaskonstante, $T_0$
der Siedepunkt des Lösungsmittels und $\Lambda_A$ deren modulare Verdampfungswärme ist.

\subsection{Dissoziation}
Dissoziation bedeutet, dass sich ein Stoff in kleinere Teilchen 
oder Ionen aufspaltet; hierzu wird jedoch ein Lösungsmittel 
benötigt, das die nötige Energie aufbringen kann, um atomare 
Bindungen aufzutrennen. Wasser ist ein gutes Lösungsmittel, 
da es stark polar ist. Die Polarität von Wasser stammt von den 
starken Partialladungsunterschieden innerhalb eines Wassermoleküls. 
Diese starken Partialladungen entstehen dadurch, dass 
Wasser zum einen stark elektronegativ ist \cite{Elektronegativ} und 
zum anderen zwei freie Elektronenpaare besitzt, die zusätzlich 
negativ geladen sind. Dies führt dazu, dass die Elektronen in den 
beiden Wasserstoffbindungen zum Sauerstoff gezogen werden und somit 
eine negative Seite am Sauerstoff und eine positive Seite bei den 
Wasserstoffatomen entsteht. Die erweiterte Formel lautet somit:


\begin{equation}
    \Delta T = \frac{m_B}{m_A M_B}K(1+\alpha(z-1))
    \label{eq:alpha}
\end{equation}