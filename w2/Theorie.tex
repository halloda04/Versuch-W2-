\section{Theoretische Grundlagen}

\subsection{Aggregatzustände}
Die drei Aggregatzustände die wir in unserem täglichen Leben 
beobachten können sind, Gasförmig, Flüssig und fest. Im 
Gasförmigen Zustand haben die Teilchen eines Stoffes genug 
Energie um die Wechselwirkungen die sie untereinander haben 
zu durchbrechen, dadurch können die Teilchen sich frei im 
Raum bewegen. Wenn nun Energie von diesen Teilchen entfernt 
wird, zum Beispiel durch abkühlen, werden diese Teilchen 
erst zu einer flüssigkeit, wenn nun weiter Energie entfernt 
wird gefriert diese Flüssigkeit und wird zu einem Feststoff. 
Teilchen in einer Flüssigkeit haben noch genug Energie um 
sich zu bewegen aber nicht mehr genug Energie um sich 
komplett frei zu bewegen. In einem Feststoff wiederum ist so 
wenig Energie dass die Teilchen oft in einem Gitter 
angeordnet vorzufinden sind und sich nur noch minimal durch 
schwingungen bewegen können.
Phasendiagramm:
In diesem Diagramm sind die Übergänge dieser drei 
Aggregatzustände in Abhängigkeit von ihrer 
Temperatur und dem Druck des betrachteten Volumen zu sehen. 
\subsection{Dampfdruck}
Die Dampfdruck Kurve bezeichnet den Übergang vom Flüssigen 
in das Gasförmige. Genau auf der Dampfdruckkurve befindet 
sich ein Thermodynamisches Gleichgewicht zwischen Teilchen 
die aus der Flüssigkeit in das Gasförmige übergehen und 
umgekehrt
\subsection{Dissoziation}