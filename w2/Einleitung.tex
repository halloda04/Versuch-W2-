\section{Einleitung}

Die Siedepunkt erhöhung ist ein wichitges Konzept 
der aktuellen Welt, sie findet sich in Anwendungen 
des täglichen Lebens, in der Technik und in der 
Wissenschaft. Im täglichen Leben wird die 
Siedepunkt erhöhung zum Beispiel dazu verwendet 
Vereiste Straßen zu enteisen, hierzu wird einfach 
nur Kochsalz auf das Eis gestreut, dadurch steigt 
der gefrier Punkt des Wassers, dadurch kann es 
einfach von der Straße abfließen. In der Technik 
kann genau der Gegenteilige Prozess zum Einsatz 
kommen, zum beispiel bei der Anwendung von 
Wärmetransport hierbei wird der Siedepunkt durch 
einen niedrigeren Druck nach unten gesetzt um bei 
kleinen Temperatur Änderungen schon eine verdampung 
zu stande zu bringen und somit viel Wärme Energie 
von einem kleinen Ort abtransportieren zu können. 
Beispielsweise bei Rechenstarken Microchips. In der 
Wissenschaft kann die Siedepunktserhöhung dazu 
verwendet werden die Stoffmenge zu ermitteln die in 
einem Lösungsmittel gelöst wurde. Der durchzuführende 
Versuch ist dem Bereich der Wärmelehre zuzuordnen.