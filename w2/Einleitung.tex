\section{Einleitung}

Die Siedepunktserhöhung ist ein wichtiges Konzept 
der aktuellen Welt; sie findet sich in Anwendungen 
des täglichen Lebens, in der Technik und in der 
Wissenschaft. Im täglichen Leben wird die 
Siedepunktserhöhung zum Beispiel dazu verwendet, 
vereiste Straßen zu enteisen. Hierzu wird einfach 
Kochsalz auf das Eis gestreut; dadurch sinkt 
der Gefrierpunkt des Wassers, sodass es 
einfach von der Straße abfließen kann. In der Technik 
kann genau der gegenteilige Prozess zum Einsatz 
kommen. Zum Beispiel wird beim Wärmetransport 
der Siedepunkt durch einen niedrigeren Druck 
herabgesetzt, um schon bei kleinen Temperaturänderungen 
eine Verdampfung zu erzeugen und somit viel Wärmeenergie 
von einem kleinen Ort abtransportieren zu können, 
beispielsweise bei rechenstarken Mikrochips. In der 
Wissenschaft kann die Siedepunktserhöhung dazu 
verwendet werden, die Stoffmenge zu ermitteln, die in 
einem Lösungsmittel gelöst wurde. Der durchzuführende 
Versuch ist dem Bereich der Wärmelehre zuzuordnen.
