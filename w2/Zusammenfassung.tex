\section{Zusammenfassung}
Die Teilaufgaben 1 und 2 konnten erfolgreich behandelt werden. Auch wenn in Teilaufgabe 2 der errechnete Wert nicht mit einem Literaturwert verglichen werden konnte, scheint das Ergebnis sinnvoll. In Teilaufgabe 3 stimmt der errechnete Wert für die 20\%-ige Lösung definitiv nicht mit der Realität überein. Dies ist aufgrund von nicht mehr gültigen Näherungen sowie ungenauen Messwerten und potenziell noch weiteren Fehlern der Fall. Insgesamt konnte das physikalische Phänomen der Siedepunkterhöhung aber schön festgestellt werden. Auch wurde bestätigt, dass mit höherer Konzentration der zugeführten Substanzen der Siedepunkt ebenfalls gestiegen ist.